UPDATED---\today.
Modern climbing assistance systems allow defining routes using external devices.
This makes the training process unintuitive.
We believe such a system should be able to autonomously record and identify routes by observing the climber.
The difficulty here is to provide a reliable detection of the holds grabbed.
We tackle this issue using a hybrid approach based on visual analysis with a measurement of the muscle activity.
In particular, this works as follows:
First, we utilize the data given by Beta-Cube to match the position of the hand or foot with the closest hold.
Second, we use the Myo muscle tracker (or similar) to determine if the hold was actually grabbed.