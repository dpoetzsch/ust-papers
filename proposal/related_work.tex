There are currently several approaches developed or in development that partially solve the task of autonomous climbing route recognition and route definition.
However, all current approaches have in common that they are missing essential features in order to be deployed as a fully functional system or are infeasible. 
The following subsections will highlight the most relevant tracking and detection methods and will depict their advantages and restrictions. 

\subsection{Inertia Based Systems}
The ClimSense system \cite{pub7648, Kosmalla:2015:CAC:2702123.2702311} uses data of recorded ascents in order to recognize already climbed routes. 
The data recording is performed by using wrist-worn inertia measurement units (IMUs).
As already mentioned this system relies on recorded data, but we do want the restriction to only be able to classify known routes. We aim at identifying new routes as well. 

\subsection{Optical Systems}
betaCube \cite{pub8245, Wiehr:2016:BET:2851581.2892393} uses an accompanying Android application which has to be used in order to create a route. 
Furthermore skeleton tracking of the Kinect camera is used in order to detect the climber during his ascent, this however introduces some disadvantages like occlusion and grabbing of occluded holds. 
Furthermore the optical technologies are not able to detect with certainty of a grab is actually grabbed.
The approach described in the paper 'Augmented Climbing: interacting with projected graphics on a climbing wall' \cite{Kajastila:2014:ACI:2611780.2581139, Kajastila:2014:ACI:2559206.2581139} is because of it's optical nature not occlusion resistant, too. 
Therefore it has the same resitrictions as the betaCube.

\subsection{Muscular Tracking}
The Myo Gesture Control Armband \cite{Myo:Online} consists of eight electromyographic (EMG) sensors that are able to measure myostatic muscle tension. 
Additional the Myo is equipped with an inertial measurement unit (IMU). 
A similar system equipped with additional pressure sensors is the EMPress \cite{McIntosh:2016:EPH:2858036.2858093}. 
Both systems allow the detection of hold grabbing but are limited in the concrete positioning of hand and feet on the wall.

\subsection{Smart Holds}
Smart holds are holds equipped with sensors that allow the acquisition of additional data during the ascent of a climber. 
This allows for a very precise route tracking using e.g. RFID identification of individual holds and/or force measurement \cite{Kistler:Online, Lechner:Online}. 
\todo{find additional/better references}
However, this approach has the tremendous disadvantage that all holds of a gym have to be replaced in order to provide the functionality of autonomous climbing route recognition and route definition to all athletes. 
This is rather expensive and therefore infeasible approach.  

\iffalse
	\todo{Describe the currently existing technologies and their advantages and disadvantages. --> MARC} \\ 

	\todo{also mention the overall approach that the position tracking and matches the aggregated data against a database of existing routes}

	testciting \cite{McIntosh:2016:EPH:2858036.2858093}
\fi