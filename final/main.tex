\documentclass{sigchi-ext}
% Please be sure that you have the dependencies (i.e., additional
% LaTeX packages) to compile this example.
\usepackage[T1]{fontenc}
\usepackage{textcomp}
\usepackage[scaled=.92]{helvet} % for proper fonts
\usepackage{graphicx} % for EPS use the graphics package instead
\usepackage{balance}  % for useful for balancing the last columns
\usepackage{booktabs} % for pretty table rules
\usepackage{ccicons}  % for Creative Commons citation icons
\usepackage{ragged2e} % for tighter hyphenation

% \usepackage{marginnote} \usepackage[shortlabels]{enumitem}
% \usepackage{paralist}

\newcommand{\todo}[1]{{\color{red}#1}}

%% EXAMPLE BEGIN -- HOW TO OVERRIDE THE DEFAULT COPYRIGHT STRIP --
% \copyrightinfo{Permission to make digital or hard copies of all or
% part of this work for personal or classroom use is granted without
% fee provided that copies are not made or distributed for profit or
% commercial advantage and that copies bear this notice and the full
% citation on the first page. Copyrights for components of this work
% owned by others than ACM must be honored. Abstracting with credit is
% permitted. To copy otherwise, or republish, to post on servers or to
% redistribute to lists, requires prior specific permission and/or a
% fee. Request permissions from permissions@acm.org.\\
% {\emph{CHI'14}}, April 26--May 1, 2014, Toronto, Canada. \\
% Copyright \copyright~2014 ACM ISBN/14/04...\$15.00. \\
% DOI string from ACM form confirmation}
%% EXAMPLE END

\title{Enhancing Autonomous Climbing Route Recognition}

\numberofauthors{4}
% Notice how author names are alternately typesetted to appear ordered
% in 2-column format; i.e., the first 4 autors on the first column and
% the other 4 auhors on the second column. Actually, it's up to you to
% strictly adhere to this author notation.
\author{%
  \alignauthor{%
    \textbf{Anna Krajuskina}\\
    \affaddr{Saarland University} \\
    \affaddr{Department of Computer Science} \\
    \affaddr{Saarbruecken, Germany} \\
    \affaddr{anna.kraj@mail.ru} }\alignauthor{%
    \textbf{David Poetzsch-Heffter}\\
    \affaddr{Saarland University} \\
    \affaddr{Department of Computer Science} \\
    \affaddr{Saarbruecken, Germany} \\
    \affaddr{davidpoetzsch@posteo.de} } \vfil \alignauthor{%
    \textbf{Felix Philipp Felten}\\
    \affaddr{Saarland University} \\
    \affaddr{Department of Computer Science} \\
    \affaddr{Saarbruecken, Germany} \\
    \affaddr{FPF1993@googlemail.com} }\alignauthor{%
    \textbf{Marc Schweig}\\
    \affaddr{Saarland University} \\
    \affaddr{Department of Computer Science} \\
    \affaddr{Saarbruecken, Germany} \\
    \affaddr{marc\_schweig@gmx.de} } }

% Paper metadata (use plain text, for PDF inclusion and later
% re-using, if desired)
\def\plaintitle{Enhancing Autonomous Climbing Route Recognition} 
\def\plainauthor{Anna Krajuskina, David Poetzsch-Heffter, Felix Philipp Felten,
 Marc Schweig}
\def\plainkeywords{Image Processing; Visual Computing; Ubiquitous Sports Technologies; Wearable Computing; Activity Tracking; Climbing}
\def\plaingeneralterms{Documentation, Standardization}

%% Set up our PDF with metadata
\hypersetup{%
  pdftitle={\plaintitle}, pdfauthor={\plainauthor},
  pdfkeywords={\plainkeywords}, }

% \reversemarginpar%

\begin{document}

\maketitle

% Uncomment to disable hyphenation (not recommended)
% https://twitter.com/anjirokhan/status/546046683331973120
\RaggedRight{} 

% Do not change the page size or page settings.
\begin{abstract}
    UPDATED---\today.
Modern climbing assistance systems allow defining routes using external devices.
This makes the training process unintuitive.
We believe such a system should be able to autonomously record and identify routes by observing the climber.
The difficulty here is to provide a reliable detection of the holds grabbed.
We tackle this issue using a hybrid approach based on visual analysis with a measurement of the muscle activity.
In particular, this works as follows:
First, we utilize the data given by Beta-Cube to match the position of the hand or foot with the closest hold.
Second, we use the Myo muscle tracker (or similar) to determine if the hold was actually grabbed.
\end{abstract}

\keywords{\plainkeywords}

\category{H.1.2}{User/Machine Systems}{Human factors}
\category{H.5.m}{Information interfaces and presentation (e.g., HCI)}{Miscellaneous}

\section{Introduction}
	\label{sec:intro}
    Recognizing climbing routes is crucial for many climbing-associated applications:
It allows to intuitively define new routes by climbing them and track the climbed routes in order to measure training performance.
In a second step the gathered statistical data can be used e.g.\ for automatic difficulty estimation.

Climbing route recognition that allows recording of new routes heavily depends on reliably detecting which individual holds were used by the climber.
Several approaches exist, including smart holds \cite{Kistler:Online, Lechner:Online} or visual detection \cite{pub8245, Wiehr:2016:BET:2851581.2892393, Kajastila:2014:ACI:2611780.2581139, Kajastila:2014:ACI:2559206.2581139}.
However, smart holds are rather expensive and visual detection suffers from reliability issues \cite{pub8245, Wiehr:2016:BET:2851581.2892393, Kajastila:2014:ACI:2611780.2581139, Kajastila:2014:ACI:2559206.2581139}.
To overcome these issues we propose a hybrid approach that combines visual detection with a muscle tracker.
The muscle tracker allows us to identify if a hold was grabbed by detecting if the climber flexed the muscles in her arms.
The visual detection system matches the position of the hand with the closest hold on the wall.

As a result, our technology not only allows to distinguish existing routes but also provides means to define new ones just by climbing them.

In the following report we will first give an overview over the existing approaches in route recognition.
Then, we discuss the details of our implementation of the proposed approach.
We conclude by evaluating our approach and giving an outlook for future work.

\section{Related Work}
    \label{sec:relatedwork}
    There are currently several approaches developed or in development that partially solve the task of autonomous climbing route recognition and route definition.
However, all current approaches have in common that they are missing essential features in order to be deployed as a fully functional system or are infeasible. 
The following subsections will highlight the most relevant tracking and detection methods and will depict their advantages and restrictions. 

\subsection{Inertia Based Systems}
The ClimbSense system \cite{pub7648, Kosmalla:2015:CAC:2702123.2702311} uses data of recorded ascents in order to recognize already defined routes. 
The data recording is performed by using wrist-worn inertia measurement units (IMUs).
However, since ClimbSense relies on pre-recorded data, it cannot solve the task of route definition.


\subsection{Optical Systems}
betaCube \cite{pub8245, Wiehr:2016:BET:2851581.2892393} uses an accompanying Android application which has to be used in order to create a route. 
Furthermore, skeleton tracking of the Kinect camera is applied in order to detect the climber during his ascent.
A similar optical approach is described in the paper 'Augmented Climbing: interacting with projected graphics on a climbing wall' \cite{Kajastila:2014:ACI:2611780.2581139, Kajastila:2014:ACI:2559206.2581139}.
However, both optical systems suffer from the inability to detect occluded holds. 
In general, optical technologies are currently not able to detect with certainty if a hold is actually grabbed.

\subsection{Muscular Tracking}
The Myo Gesture Control Armband \cite{Myo:Online} consists of eight electromyographic (EMG) sensors that are able to measure myostatic muscle tension. 
Additional the Myo is equipped with an inertial measurement unit (IMU). 
A similar system equipped with additional pressure sensors is the EMPress \cite{McIntosh:2016:EPH:2858036.2858093}. 
Both systems allow the detection of hold grabbing but are not able to determine the position of a climber's hands and feet.

\subsection{Smart Holds}
Smart holds are holds equipped with sensors that allow the acquisition of additional data during the ascent of a climber. 
This allows for a very precise route tracking using e.g. RFID identification of individual holds and/or force measurement \cite{Kistler:Online, Lechner:Online}. 
However, this approach has the tremendous disadvantage that all holds of a gym have to be replaced in order to provide the functionality of autonomous climbing route recognition and route definition to all athletes. 
This is a rather expensive and therefore infeasible approach.  

\iffalse
	\todo{Describe the currently existing technologies and their advantages and disadvantages. --> MARC} \\ 

	\todo{also mention the overall approach that the position tracking and matches the aggregated data against a database of existing routes}

	testciting \cite{McIntosh:2016:EPH:2858036.2858093}
\fi
\section{Implementation}
    \label{sec:implementation}
    Our system combines image analysis with muscle tracking to provide the technology that will allow climbers to define routes simply by climbing them.
In particular, our system is constructed as follows:
During the climber's ascent, the muscle data of her arms is constantly monitored by our \emph{Myo server}.
Additionally, the climber is filmed by a camera integrated into the \emph{betaCube}.
Now, when the climber grabs a hold the following happens:
The Myo server detects a specific increase in muscle activity and notifies the betaCube that a hold was grabbed.
The betaCube then extracts the position of the climber's hands (and potentially feet) from the visual data provided by its camera.
That position is matched with the previously stored positions of the holds on the wall.
If a hold is found within a certain radius around the position of the climbers hand the betaCube will store that hold as part of the route.
Finally, the betaCube provides visual feedback to the climber by highlighting the found hold.

In the following sub sections we will have an in-depth look into the core components of our system.

\subsection{Grab detection via muscle tracking}
Our system uses the Myo tracker for measuring the climbers muscle activity.
The Myo tracker is a wearable for the forearm that was primarily developed to detect gestures and poses of the hand to control a computer or smartphone.
It has myoelectric sensors that spot the small electric pulses the body uses to activate the muscles.
Furthermore, it is able to recognize in which angle the arm is held using an integrated inertia measurement unit (IMU).

The muscle detection part of our system is called the \emph{Myo server}.
It is a stand-alone server that is connected both to the Myo wearables via bluetooth and to the betaCube via a TCP connection.

At run time, the server waits for specific values that occur at a few of the muscle sensors every time the climber starts to grab a hold.
Then the system enters a pending phase where it continues to listen for the muscle data.
Each received data-set is summed up and compared to a threshold.
If most of the received values exceed that threshold, the system signals a grab and transitions to the holding state.
The latter is terminated when the muscle data starts to continuously drop below the threshold, returning the system to the initial waiting phase.
All threshold values were determined empirically.

The grab signal of the Myo server consists of a short JSON message containing the current time and the extremity where the grab was detected (i.e., left or right hand) to the betaCube.

\subsection{Hand to hold matching via visual analysis}
For the image analysis we rely on the betaCube technology.
The betaCube is a climb tracking system consisting of a kinect camera for visual tracking and an integrated projector for projecting things to the wall.
Additionally the betaCube already has the locations of all holds on the wall stored.

In order to facilitate the extremity recognition, the climber wears colored bands on each limb.
Each band is of different color which are predefined in the program.

Our system tightly integrates with the betaCube software.
In particular we extended the software such that it connects to the Myo server at start up and listens for grab events.
When a grab event occurs, our system retrieves the current picture of the climber from the Kinect camera.
The program then attempts to categorize each pixel into the predefined color groups.
The color groups are defined by the difference between their RGB components.
If the pixel was categorized successfully and the color group is the one mapped to certain limb, the coordinates of the pixel are returned.
Otherwise, the pixel is skipped.
Finally, the system checks for a hold in the neighborhood of the found pixel.
If such a hold is found, it is marked as grabbed.

\subsection{Getting rid of the colored wristbands}
We also worked on a more sophisticated approach to find the hand of the climber in the picture.
In this second approach the climber does not have to wear a colored wristband.
Instead, the Myo wearables were colored.
The Myo server was extended to additionally calculate the x- and y-offset of the hand from the position of the Myo wearable.
We did so by using the rotational data provided by the Myo's integrated IMU and an estimate of the length of the climbers forearm.
The betaCube then adjusts the found color position of the Myo using those offsets.

\subsection{Deletion of previously grabbed holds}

In order to delete an already defined hold of the route, we use the gesture detection capabilities of the Myo SDK.
When the climber performs such a gesture, the Myo Server sends a deletion signal to the betaCube.
The latter then removes the last added hold.
However, evaluation showed that gesture recognition by the Myo SDK is not reliable.

\section{Evaluation}
	\label{sec:evaluation}
    Field studies showed that our system was able to reliably detect grabbed hand holds.
We also built the infrastructure to detect foot holds, however the Myo wearable physically does not fit legs.
Nevertheless, our system architecture would easily allow to plug in detection of foot holds using pressure sensors in the climber's shoes.
The betaCube is already equipped to find feet on the wall as well and match them with the closest foot holds.

We believe that the proposed technology could be applied in a bouldering gym with a so-called moon board that allows for free training.
Climbers could easily define routes for themselves and make them available to others.
That our technology does not rely on external electronic devices is a bonus here because climbers often use chalk which might keep them from using standard electronic devices like mobile phones.

Compared to our initial proposal we fulfilled all specified must-haves.
It is worth mentioning that in the proposal we assumed that the betaCube system is already able to provide us with location data of hand and feet.
As this was not the case, we built this part from scratch.
Additionally, we extended our approach with a deletion capability of the previously grabbed holds in order to make it useful.

\section{Conclusion \& Future Work}
	\label{sec:conclusion}
    In this paper we proposed a system that is able to autonomously recognize climbing routes.
We solved the task of detecting grabbed holds by a hybrid approach:
Muscle tracking allowed us to identify \emph{if} a hold was grabbed, Visual analysis decided \emph{which} hold was grabbed.
Field studies showed that our system was reliably detecting grabbed hand holds.

For the future, an important step would be to integrate foot hold detection into the system, for example using pressure sensors in the climber's shoes.
As mentioned before, we already paved the way for this extension.

Another important improvement would be to stabilize the offset calculation that would enable us to get rid of the colored wristbands.
Also, the colors could be replaced, for example by an infrared signal.
betaCube's kinect camera already provides means to detect such a signal.

Finally, a precise statistical evaluation of the system with different climbers and conditions is still to be done.

\makeatletter
\def\@copyrightspace{\relax}
\makeatother

\iffalse
    \section{Introduction}
    This format is to be used for submissions that are published in the
    conference publications. We wish to give this volume a consistent,
    high-quality appearance. We therefore ask that authors follow some
    simple guidelines. In essence, you should format your paper exactly
    like this document. The easiest way to do this is to replace the
    content with your own material.
    
    \section{ACM Copyrights \& Permission}
    Accepted extended abstracts and papers will be distributed in the
    Conference Publications. They will also be placed in the ACM Digital
    Library, where they will remain accessible to thousands of researchers
    and practitioners worldwide. To view the ACM's copyright and
    permissions policy, see:
    \url{http://www.acm.org/publications/policies/copyright_policy}.
    
    \marginpar{%
      \vspace{-45pt} \fbox{%
        \begin{minipage}{0.925\marginparwidth}
          \textbf{Good Utilization of the Side Bar} \\
          \vspace{1pc} \textbf{Preparation:} Do not change the margin
          dimensions and do not flow the margin text to the
          next page. \\
          \vspace{1pc} \textbf{Materials:} The margin box must not intrude
          or overflow into the header or the footer, or the gutter space
          between the margin paragraph and the main left column. The text
          in this text box should remain the same size as the body
          text. Use the \texttt{{\textbackslash}vspace{}} command to set
          the margin
          note's position. \\
          \vspace{1pc} \textbf{Images \& Figures:} Practically anything
          can be put in the margin if it fits. Use the
          \texttt{{\textbackslash}marginparwidth} constant to set the
          width of the figure, table, minipage, or whatever you are trying
          to fit in this skinny space.
        \end{minipage}}\label{sec:sidebar} }
    
    \section{Page Size}
    All SIGCHI submissions should be US letter (8.5 $\times$ 11
    inches). US Letter is the standard option used by this \LaTeX\
    template.
    
    \section{Text Formatting}
    Please use an 8.5-point Verdana font, or other sans serifs font as
    close as possible in appearance to Verdana in which these guidelines
    have been set. Arial 9-point font is a reasonable substitute for
    Verdana as it has a similar x-height. Please use serif or
    non-proportional fonts only for special purposes, such as
    distinguishing \texttt{source code} text.
    
    \subsubsection{Text styles}
    The \LaTeX\ template facilitates text formatting for normal (for body
    text); heading 1, heading 2, heading 3; bullet list; numbered list;
    caption; annotation (for notes in the narrow left margin); and
    references (for bibliographic entries). Additionally, here is an
    example of footnoted\footnote{Use footnotes sparingly, if at all.}
    text. As stated in the footnote, footnotes should rarely be used.
    
    \begin{figure}
      \includegraphics[width=0.9\columnwidth]{figures/sigchi-logo}
      \caption{Insert a caption below each figure.}~\label{fig:sample}
    \end{figure}
    
    \subsection{Language, style, and content}
    The written and spoken language of SIGCHI is English. Spelling and
    punctuation may use any dialect of English (e.g., British, Canadian,
    US, etc.) provided this is done consistently. Hyphenation is
    optional. To ensure suitability for an international audience, please
    pay attention to the following:
    
    \begin{table}
      \centering
      \begin{tabular}{r c c}
        % \toprule
        & \multicolumn{2}{c}{\small{\textbf{Caption}}} \\
        \cmidrule(r){2-3}
        {\small\textbf{Objects}}
        & {\small \textit{Pre-2002}}
        & {\small \textit{Current}} \\
        \midrule
        Tables & Above & Below \\
        Figures & Below & Below \\
        % \bottomrule
      \end{tabular}
      \caption{Table captions should be placed below the table. Minimize use of
        unnecessary table lines.}~\label{tab:table1}
    \end{table}
    
    \begin{itemize}\compresslist%
    \item Write in a straightforward style. Use simple sentence
      structure. Try to avoid long sentences and complex sentence
      structures. Use semicolons carefully.
    \item Use common and basic vocabulary (e.g., use the word ``unusual''
      rather than the word ``arcane'').
    \item Briefly define or explain all technical terms. The terminology
      common to your practice/discipline may be different in other design
      practices/disciplines.
    \item Spell out all acronyms the first time they are used in your
      text. For example, ``World Wide Web (WWW)''.
    \item Explain local references (e.g., not everyone knows all city
      names in a particular country).
    \item Explain ``insider'' comments. Ensure that your whole audience
      understands any reference whose meaning you do not describe (e.g.,
      do not assume that everyone has used a Macintosh or a particular
      application).
    \item Explain colloquial language and puns. Understanding phrases like
      ``red herring'' requires a cultural knowledge of English. Humor and
      irony are difficult to translate.
    \item Use unambiguous forms for culturally localized concepts, such as
      times, dates, currencies, and numbers (e.g., ``1-5- 97'' or
      ``5/1/97'' may mean 5 January or 1 May, and ``seven o'clock'' may
      mean 7:00 am or 19:00). For currencies, indicate equivalences:
      ``Participants were paid {\fontfamily{txr}\selectfont \textwon}
      25,000, or roughly US \$22.''
    \item Be careful with the use of gender-specific pronouns (he, she)
      and other gender-specific words (chairman, manpower,
      man-months). Use inclusive language (e.g., she or he, they, chair,
      staff, staff-hours, person-years) that is gender-neutral. If
      necessary, you may be able to use ``he'' and ``she'' in alternating
      sentences, so that the two genders occur equally
      often~\cite{Schwartz:1995:GBF}.
    \end{itemize}
    
    \marginpar{\vspace{5pc}So long as you don't type outside the right
      margin or bleed into the gutter, it's okay to put annotations over
      here on the left, too. You'll have to manually align the margin
      paragraphs to your \LaTeX\ floats using the
      \texttt{{\textbackslash}vspace{}} command.}
    
    % \begin{figure}
    %   \includegraphics[width=.9\columnwidth]{figures/ea-figure2}
    %   \caption{If your figure has a light background, you can set its
    %     outline to light gray, like this, to make a box around
    %     it.}\label{fig:bats}
    % \end{figure}
    
    \begin{marginfigure}[-45pc]
      \begin{minipage}{\marginparwidth}
        \centering
        \includegraphics[width=0.9\marginparwidth]{figures/cats}
        \caption{In this image, the cats are tessellated within a square
          frame. Images should also have captions and be within the
          boundaries of the sidebar on page~\pageref{sec:sidebar}. Photo:
          \cczero.}~\label{fig:marginfig}
      \end{minipage}
    \end{marginfigure}
    
    \section{Figures}
    The examples on this and following pages should help you get a feel
    for how screen-shots and other figures should be placed in the
    template. Your document may use color figures (see
    Figures~\ref{fig:sample}), which are included in the page limit; the
    figures must be usable when printed in black and white. You can use
    the \texttt{\marginpar} command to insert figures in the (left) margin
    of the document (see Figure~\ref{fig:marginfig}). Finally, be sure to
    make images large enough so the important details are legible and
    clear (see Figure~\ref{fig:cats}).
    
    \section{Tables}
    You man use tables inline with the text (see Table~\ref{tab:table1})
    or within the margin as shown in Table~\ref{tab:table2}. Try to
    minimize the use of lines (especially vertical lines). \LaTeX\ will
    set the table font and captions sizes correctly; the latter must
    remain unchanged.
    
    \section{Accessibility}
    The Executive Council of SIGCHI has committed to making SIGCHI
    conferences more inclusive for researchers, practitioners, and
    educators with disabilities. As a part of this goal, the all authors
    are asked to work on improving the accessibility of their
    submissions. Specifically, we encourage authors to carry out the
    following five steps:
    \begin{itemize}\compresslist%
    \item Add alternative text to all figures
    \item Mark table headings
    \item Generate a tagged PDF
    \item Verify the default language
    \item Set the tab order to ``Use Document Structure''
    \end{itemize}
    
    For links to instructions and resources, please see:
    \url{http://chi2016.acm.org/accessibility}
    
    Unfortunately good tools do not yet exist to create tagged PDF files
    from Latex. \LaTeX\ users will need to carry out all of the above
    steps in the PDF directly using Adobe Acrobat, after the PDF has been
    generated.
    
    For more information and links to instructions and resources, please
    see:
    \url{http://chi2016.acm.org/accessibility}.
    
    \section{Producing and Testing PDF Files}
    We recommend that you produce a PDF version of your submission well
    before the final deadline. Your PDF file must be ACM DL Compliant and
    meet stated requirements,
    \url{http://www.sheridanprinting.com/typedept/ACM-distilling-settings.htm}.
    
    \begin{figure*}
      \centering
      \includegraphics[width=1.4\columnwidth]{figures/map}
      \caption{In this image, the map maximizes use of space. You can make
        figures as wide as you need, up to a maximum of the full width of
        both columns. Note that \LaTeX\ tends to render large figures on a
        dedicated page. Image: \ccbynd~ayman on Flickr.}~\label{fig:cats}
    \end{figure*}
    
    \begin{margintable}[1pc]
      \begin{minipage}{\marginparwidth}
        \centering
        \begin{tabular}{r r r}
          & {\small \textbf{First}}
          & {\small \textbf{Second}} \\
          \toprule
          Child & 22 & 44 \\
          Adult & 22 & 16 \\
          \midrule
          Gene & 22 & 11 \\
          Cliff & 34 & 22 \\
          \bottomrule
        \end{tabular}
        \caption{A simple narrow table in the left margin
          space.}~\label{tab:table2}
      \end{minipage}
    \end{margintable}
    Test your PDF file by viewing or printing it with the same software we
    will use when we receive it, Adobe Acrobat Reader Version 10. This is
    widely available at no cost. Note that most
    reviewers will use a North American/European version of Acrobat
    reader, so please check your PDF accordingly.
    
    \section{Acknowledgements}
    We thank all the volunteers, publications support, staff, and authors
    who wrote and provided helpful comments on previous versions of this
    document. As well authors 1, 2, and 3 gratefully acknowledge the grant
    from NSF (\#1234--2222--ABC). Author 4 for example may want to
    acknowledge a supervisor/manager from their original employer. This
    whole paragraph is just for example. Some of the references cited in
    this paper are included for illustrative purposes only.
    
    \section{References Format}
    Your references should be published materials accessible to the
    public. Internal technical reports may be cited only if they are
    easily accessible (i.e., you provide the address for obtaining the
    report within your citation) and may be obtained by any reader for a
    nominal fee. Proprietary information may not be cited. Private
    communications should be acknowledged in the main text, not referenced
    (e.g., ``[Golovchinsky, personal communication]'').
    
    Use a numbered list of references at the end of the article, ordered
    alphabetically by first author, and referenced by numbers in
    brackets~\cite{ethics,Klemmer:2002:WSC:503376.503378}. For papers from
    conference proceedings, include the title of the paper and an
    abbreviated name of the conference (e.g., for Interact 2003
    proceedings, use Proc.\ Interact 2003). Do not include the location of
    the conference or the exact date; do include the page numbers if
    available. See the examples of citations at the end of this document
    and in the accompanying \texttt{BibTeX} document.
    
    References \textit{must be the same font size as other body
      text}. References should be in alphabetical order by last name of
    first author. Example reference formatting for individual journal
    articles~\cite{ethics}, articles in conference
    proceedings~\cite{Klemmer:2002:WSC:503376.503378},
    books~\cite{Schwartz:1995:GBF}, theses~\cite{sutherland:sketchpad},
    book chapters~\cite{winner:politics}, a journal issue~\cite{kaye:puc},
    websites~\cite{acm_categories,cavender:writing},
    tweets~\cite{CHINOSAUR:venue}, patents~\cite{heilig:sensorama}, and
    online videos~\cite{psy:gangnam} is given here. This formatting is a
    slightly abbreviated version of the format automatically generated by
    the ACM Digital Library (\url{http://dl.acm.org}) as ``ACM Ref''. More
    details of reference formatting are available at:
    \url{http://www.acm.org/publications/submissions/latex_style}.
\fi

\balance{} 

% \bibliographystyle{ACM-Reference-Format-Journals}
\bibliographystyle{SIGCHI-Reference-Format.bst}
% \bibliographystyle{acm}
\bibliography{references}

\end{document}

%%% Local Variables:
%%% mode: latex
%%% TeX-master: t
%%% End:
