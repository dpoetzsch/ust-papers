There are currently several approaches developed or in development that partially solve the task of autonomous climbing route recognition and route definition.
However, all current approaches have in common that they are missing essential features in order to be deployed as a fully functional system or are infeasible. 
The following subsections will highlight the most relevant tracking and detection methods and will depict their advantages and restrictions. 

\subsection{Inertia Based Systems}
The ClimbSense system \cite{pub7648, Kosmalla:2015:CAC:2702123.2702311} uses data of recorded ascents in order to recognize already defined routes. 
The data recording is performed by using wrist-worn inertia measurement units (IMUs).
However, since ClimbSense relies on pre-recorded data, it cannot solve the task of route definition.


\subsection{Optical Systems}
betaCube \cite{pub8245, Wiehr:2016:BET:2851581.2892393} uses an accompanying Android application which has to be used in order to create a route. 
Furthermore, skeleton tracking of the Kinect camera is applied in order to detect the climber during his ascent.
A similar optical approach is described in the paper 'Augmented Climbing: interacting with projected graphics on a climbing wall' \cite{Kajastila:2014:ACI:2611780.2581139, Kajastila:2014:ACI:2559206.2581139}.
However, both optical systems suffer from the inability to detect occluded holds. 
In general, optical technologies are currently not able to detect with certainty if a hold is actually grabbed.

\subsection{Muscular Tracking}
The Myo Gesture Control Armband \cite{Myo:Online} consists of eight electromyographic (EMG) sensors that are able to measure myostatic muscle tension. 
Additional the Myo is equipped with an inertial measurement unit (IMU). 
A similar system equipped with additional pressure sensors is the EMPress \cite{McIntosh:2016:EPH:2858036.2858093}. 
Both systems allow the detection of hold grabbing but are not able to determine the position of a climber's hands and feet.

\subsection{Smart Holds}
Smart holds are holds equipped with sensors that allow the acquisition of additional data during the ascent of a climber. 
This allows for a very precise route tracking using e.g. RFID identification of individual holds and/or force measurement \cite{Kistler:Online, Lechner:Online}. 
However, this approach has the tremendous disadvantage that all holds of a gym have to be replaced in order to provide the functionality of autonomous climbing route recognition and route definition to all athletes. 
This is a rather expensive and therefore infeasible approach.  

\iffalse
	\todo{Describe the currently existing technologies and their advantages and disadvantages. --> MARC} \\ 

	\todo{also mention the overall approach that the position tracking and matches the aggregated data against a database of existing routes}

	testciting \cite{McIntosh:2016:EPH:2858036.2858093}
\fi